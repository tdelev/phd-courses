%%This is a very basic essay template based on letter class.
\documentclass[12pt,a4paper]{article}
\usepackage{ucs}
\usepackage[T2A]{fontenc}
\usepackage[utf8]{inputenc}
\usepackage[english,bulgarian]{babel}
\usepackage[margin=1in]{geometry}
%\usepackage[bindingoffset=2cm,centering,includeheadfoot,margin=1in]{geometry}
\usepackage{graphicx}


\begin{document}

\thispagestyle{empty}
\input{essay_title_page.tex}
\vfill
\emph{„Живеј како да ти е последен ден. Учи како кога би живеел вечно.“} - Ганди
\vfill

Со успех и посветеност сум го завршил основното и средното
образование. Сум дипломирал и магистрирал на технички факултет и до овој
момент кога сум дел од докторска школа не сум имал можност или
среќа да учествувам или бидам дел од некаква форма на етичко образование.
Слушам и читам за основните начела на оваа наука, чувствувам како од секогаш
да сум знаел по нешто, сум слушал често или пак читал за најважните етички принципи и
прашања. Секоја моја одлука како да постапам во одредена ситуација, како да се
однесувам, да разликувам помеѓу добро и лошо, всушност побудувала
одредени етички прашања. Што е тоа што е добро за мене, за моите најблиски, за
поширокото опкружување? Како да постапувам во животот, кои да бидат моите цели и
кој е патот по кој ќе ги остварувам? Во текот на целиот живот се соочуваме
со овие и безброј други етички прашања.

Сепак, иако формално не сум учел за етиката и моралните вредности, имам чувство
дека од првата моја самостојна одлука, од првиот мој посериозен контакт со
околината и општеството секогаш сум наоѓал одговор на овие прашања. Од
каде и како учиме да се справуваме со овие прашања? Дали знаеме доволно за
да ја донесеме вистинската одлука, да одговориме на животните прашања? 
Дали ќе се изградиме во подобри личности ако знаеме повеќе за основните етички
прашања? И конечно, дали сите луѓе имаат право да научат повеќе за најважните
прашања во животот, за прашањата чии одговори ја формираат нивната личност, за одлуките кои
ќе им го трасираат патот до целта\ldots

Човекот еволуциски е најразвиено суштество на земјата. Постојат теории за овој 
еволуциски развој, но за развојот и оформувањето на едно човечко суштество од
неговото раѓање до смртта сме секојдневни сведоци. Овој раст, развој и
оформување на нов жител на нашата планета се одвива пред нашите очи и ние сме
неразделив дел од него. Опкружувањето кое го создаваме е всушност околината која
има најголемо влијание во создавањето на новите индивидуи. Сите наши зборови,
постапки, како и околината која ја создаваме претставуваат дел од овој
комплексен мозаик. На глобално ниво, значајни коцки од овој мозаик се
нацијата, културата и религијата во која се раѓа новиот жител. 

Ако нацијата, културата и религијата се многу значајни сегменти од животот
кој ги разработува етиката, сепак сметам дека еден од најважните сегменти кој е
многу значаен за моралното и етичкото оформување посебно на децата се семејството и фамилијата. Тоа е
микросветот во кој новите жители осознаваат за светот, го имаат првиот допир со
реалноста и практично ги добиваат одговорите на првите етички прашања.

Најголем дел од децата имаат среќа да се раѓаат во оформени
и нормални семејства. Семејства кои го достигнале еден од основните стремежи
на човекот да го продолжува своето поколение и да расте и оформува нови здрави
поколенија. Семејства кои создаваат околина исполнета со луѓето кои не сакаат,
кои се грижат за нас и постојано се обидуваат да создадат добри личности, притоа
следејќи ги сопствените етички норми и принципи. Секое дете кое израснало како
дел од здраво и нормално семејство, одговорот на првите етички прашања го добива
од своите најблиски. Тоа го слуша говорот на своите најблиски, го гледа и следи
нивното однесување. Ова за детето е првата и најзначајна етичка библија.

Меѓутоа и во светот кој денес го знаеме, колку и да е променет и напреден, не
сите деца ја имаат среќата да се родат во здрави и нормални семејства. Денес,
како никогаш порано сме сведоци на распаѓање на концептот на семејство во кој
ние сме растеле. Динамиката на животот и некои нови вредности создаваат сосема
друга околина во која некои „несреќни“ деца ќе треба да се оформуваат и учат за
своите први чекори во животот и општеството. Овде се наметнува уште едно многу
важно и суштинско прашање, дали овие деца заслужуваат подобро? Како овие нови и
идни носители на нашето општество ќе научат да се однесуваат добро, да имаат цел
во животот, да постигнуваат успех и на крај да бидат примерни и успешни граѓани?
Дали сите имаме подеднакво право да учиме и знаеме за вистинските етички вредности? 

Одговорот на овие прашање го дава смислата на постоењето на етиката. Сите сакаме
свет во кој што ќе преовладува доброто однесување, ќе има високо ниво на култура
и разбирање. Знаењето и меѓусебното почитување очекуваме да биде највисока
вредност кај луѓето. Нема да има искривени вредности и моралот кај луѓето ќе
биде на највисоко ниво. Можеме до недоглед да набројуваме за вистинските идеали
и вредности и како го замислуваме светот во кој сакаме да живееме. Впрочем и
една од целите на етиката како наука е да ни го трасира патот до вакво
општество. Тогаш зошто едноставно секој од нас да не ја учи оваа наука? Зошто да
не им се овозможи на децата уште од најмала возраст да слушаат и учат што е тоа
што е добро? Како да се биде добар? Кои се вистинските вредности и како да се
бориме за нив? Како да сакаме и да не ги повредуваме другите? Историјата
многу пати до сега била сведок дека цели империи, нации, општествени уредувања
се променети под влијание на некое учење и настојување да се индоктринира
одредена идеологија. Зошто таа доктрина да не биде учењето етика и зошто
не оставиме на времето да покаже за вистинските ефекти кои тоа ќе ги постигне
врз општеството кои сигурно нема да бидат за потценување.

Без етичко образование всушност ние и никогаш не сме учеле етика. Ако ги
запрашате луѓето што не изучувале етика, дали знаат што е тоа, повеќето ќе ви
одговорат дека знаат. Но, ако ги запрашате што е тоа етика, повеќето ќе одоворат
нешто за кое претпоставуваат дека е етика или е дел од етиката. Можеби е добро е
да се има мислење за нешто, но многу подобро е да се знае и разбира тоа, а
секогаш најдобро е да се применува, па така и за етиката. Можеби ќе дојде и тој
ден кога вистинската етика ќе ја видиме во пракса, не од одредени исклучоци
туку како норма во животот на мнозинството од луѓето. Прочуениот кинески философ
Конфучиј рекол \emph{„Слушам, па заборавам. Гледам, па паметам. Правам, па
разбирам.“} Да се надеваме дека ќе дојде децата и возрасните повеќе ќе слушаат
за етика, ќе ја гледаат околу себе и вистински ќе ја раберат.


\end{document}
