\documentclass[12pt,a4paper]{article}
\usepackage{ucs}
\usepackage[T2A]{fontenc}
\usepackage[utf8]{inputenc}
\usepackage[english,bulgarian]{babel}
\usepackage[margin=1in]{geometry}

\renewcommand\thesubsection{\alph{subsection})}

\title{Како се пишува есеј?}
\begin{document}
\maketitle


Едно од најважните белези на самостојното критичко мисле­ње е умешноста да се
изложи една теза (сопствен став), во писмена форма, и да се поткрепи со
аргументи. Најчесто, тоа се изразува во форма на есеј\footnote{Зборот есеј
потекнува од латинскиот јазик и значи обид. Како име за тектстови & размислувања
за секојдневни и филозофски прашања е употребено од францускиот филозоф Мишел де
Монтењ. Во 1580 година тој ја напишал својата книга Ессаис за карактерите и
заемните односи меѓу луѓето. Оттогаш есејот станува честа форма за изразување на
сопствените гледишта по определени прашања.}.

Во методолошката литература есеите се делат на А - мал есеј; и В - голем есеј
(она што се нарекува и семинарска работа).

Есејот (А) е краток текст (10000-15000 знаци, 5-6 страници) во кој студентот го
изложува своето гледиште за некој проблем, појава, настан, кои не само што го
заинтересира­ле него самиот, туку имаат и пошироко теориско и општест­вено
значење.

Во овој вид текст студентот треба да покаже дека знае да мисли, да расправа за
определен проблем, да има свој став по тој проблем и тоа може да го изрази во
писмена форма.

Иако се работи за сопствено гледиште и не постојат упатст­ва за тоа што ќе се
напише во еден есеј, од аспект на со­држината, сепак при пишувањето на текстот
се незаобикол­ни следниве чекори, кои секогаш не се одвиваат стриктно по
редоследот според кој ќе бидат наведени подолу (често пати некои од фазите се
преплетуваат):

\begin{enumerate}
  \item Избор на тема; 
  \item Наслов на есејот; 
  \item Композиција на текстот: вовед, тези, аргументи, заснову­вање и
  поткрепување на тезите, заклучен дел;
  \item Јазичко обликување на текстот (пишување);
  \item Употреба на цитати и документирање на есејот;
  \item Редактирање на есејот.
\end{enumerate}
 

\section{Избор на тема} 

Есејот, како секој друг пишуван текст, има своја тема. Темата на есејот може да
биде поврзана со некое прашање, однос­но проблем кој ги засега особините на
луѓето, нивните заем­ни односи, уметноста, општеството, достигањата на наука­та,
минатото, иднината итн.

Есејот е еден вид мисловен разговор на авторот со чита­телот. Токму поради тоа,
авторот треба да се концентри­ра на темата и да запише колку што е можно повеќе
пра­шања во врска со неа. Потоа, прашањата треба да се под­редат во редослед што
ќе го определи самиот автор, бидејќи кога ќе одговара на поставените прашања,
всушност тој ќе го создава текстот на својот есеј.

Есејот се гради врз основа на различни типови прашања: 

а) Прашања на кои треба да се одговори;
 
б) Прашања кои не претпоставуваат одговор, не се упатени конкретно   некому. Тоа
се реторички прашања што се користат повеќе да го навестат текот на мислите на
ав­торот и текстот да го направат подинамичен и попривлечен за читателот.

\section{Наслов на есејот} 

Насловот на есејот треба да упатува кон темата или кон основната теза. Тој мо­же
да биде формулиран како:
\begin{itemize}
  \item назив на темата: Глобализацијата и националниот иден­тит; Перспективите
  на младите во Р. Македонија; итн.
  \item да ја изразува главната теза: Глобализацијата ја обез­личува секоја
  национална култура;
  \item прашање: Дали е можен подобар свет од овој во кој живееме?
  \item некоја интересна мисла искажана од определена лич­ност: Оној што сака,
  процесот на глобализација го води, оној што не сака - го влече;
  \item поговорка: На лагата и се кратки нозете; и др. 
\end{itemize}

\section{Композиција на текстот} 

\subsection{Вовед} 

Почетокот на есејот е воведување во темата или во гле­диштето на авторот во
врска со поставениот проблем. Ав­торот избира како да го започне својот текст;

Во воведниот дел може да се укаже на она што го поттикна­ло авторот да
размислува за определена тема и да го на­пише текстот. Повод за есеј може да
биде и некоја средба, прекрасна глетка на природата, некоја мелодија или извесна
информација итн.
 
 Претставувањето на околностите во кои се јавила желба­та на авторот да се
искаже за определен проблем е осно­ва за последователни размислувања и
обопштувања. Тие водат кон обопштено тврдење за определена појава, општествен
проблем, тип на човечко однесување, вредност итн.

 Дури и кога есејот е напишан по повод на некој немил нас­тан, есејот нема за
цел да го коментира она што се случило, туку да направи обопштување за причините
и последиците на таа случка.

Во есејот, исто така, можат да бидат откриени некои подато­ци, согледби за
настани од личниот и општествениот живот, за фактите. Но целта на есејот не е
констатирање, едностав­но соопштување на информации.

Содржината на еден есеј не се исцрпува со наведувањето на фактите, тоа е
карактеристично за некој информацис­ки текст (извештај); повикувањето на фактите
во есејот е повод за размисла, не толку за самите факти, колку за значајни
проблеми кои се директно или индиректно поврзани со тие факти;

Содржината на есејот, размислувањата на оној што го пи­шува, треба да бидат
потчинети на темата и поврзани со насловот. Во спротивно, текстот нема да ја
исполни поставе­ната комуникациска задача. 

\subsection{Формулирање тези}

 Текстот на есејот се гради од тврдења кои го изразуваат гледиштето на авторот,
 неговите идеи; аргументи за потвр­дување на вистинитоста или важноста на тврдењата и
примери;
 
Откако авторот ќе ја определи насоката на своите размислу­вања раководени од
насловот, може да запише низа од тврдења (тези) врз чија основа ќе го оформиме
својот текст.

\subsection{Аргументи „за“ и „против“}

Ако насловот на есејот е формулиран како прашање, твр­дење или претпоставува
избор на становиште, авторот мо­же да направи мисловно просудување и тие
согледби да ги набележи како аргументите за поткрепа на сопствениот став, како и
можните аргументи против него.

Кога се запишани главните идеи и аргументите, авторот треба да помисли
    како ќе ги искаже и како ќе ги подреди, од што ќе произлезе неговата теза да
    биде заснована, из­ложувањето - последователно, а текстот - интересен за
    читателот.

\subsection{Логичко засновување на тезите}

Суштина на есејот е расудувањето: низа од поврзани тврде­ња по некое прашање кои
логички произлегуваат едно од друго, а како резултат на тоа се оформува
целосното гле­диште на авторот за поставениот проблем;
    
Главна карактеристика на есејот како текст-размисла е последователноста во
изложувањето. Таа го засилува впечатокот за убедливоста на изложеното мислење и
за јасноста на изразот.
    
Авторот сам ќе одлучи по каков ред ќе следуваат неговите тврдења, аргументирања.
Но насоката на мислата треба да биде таква што да не се допушти враќање кон
нешто за кое веќе е пишувано. Пос­ледователноста се изградува од различен тип
врски:
\begin{itemize}
  \item хронолошка последователност: прво се пишува за она што се случило
  порано, а потоа за она што е подоцна или обратно;
  \item причина - последица: прво се зборува за појава која претходи и обврзно предизвикува друга појава, а
потоа за резултатот што произлегува од тоа, за следството;
  \item прашање - одговор: се формулира прашање во врска со темата на есејот или
  искажаното тврдење и следно­то изложување е одговор на прашањето;
  \item пример - обопштување: се наведуваат конкретни си­туации (примери) и во
  врска со нив се прави обопштување или прво се формулира обопштување што се поткрепу­ва со примери;
  \item градација: дејствијата, својствата, заклучоците се под­редуваат според степенот на
значењето - од оние кои имаат помало значење кон позначајните или обратно.
\end{itemize}

Сепак, секогаш треба да се има на ум дека јадрото на ра­судувањето во есејот е
личното мислење, сопствената глед­на точка на авторот по избраната тема.
   
\subsection{Заклучен дел на есејот} 
 
Есејот се завршува со еден вид синтеза, обопштување на напишаното за темата. На
крајот може уште еднаш да се нагласи најзначајното од изложените тврдења, да се
истакне главната теза на авторот.

Крајот може да се оформи со не­кој афоризам, анегдота или изрека што сликовито и
ефект­но ќе го сублимира процесот на расудување, или, пак, со прашање за кое
авторот смета дека треба уште да се раз­мислува.


\section{Јазичко обликување на текстот (пишување)}

Еден добар есеј зависи и од начинот на кој е напишан: речениците треба да бидат
јасни и недвосмислени;

Не се препорачува авторот во текстот да зборува во прво лице еднина или множина,
туку да се користат безлични фор­мулации: се чини дека ...; би можело да се
заклучи ...: итн.

Еден од познатите начини за мобилизирање на мислите во определена насока е
составувањето план. Планот може да биде:

\begin{itemize}
  \item глобален - да ги содржи главните идеи;
  \item детален - освен основните идеи да се набележат и аргументите и
  примерите.
\end{itemize}
 
Есејот, кај читателот треба да предизвика чувство на ис­креност и исправност на
искажаното мислење на авторот, специфично значење на личното разбирање на некоја
висти­на.

Впечатокот за убедливост се постигнува не само со ис­користените аргументи и
докази, туку и со целосната град­ба на текстот, како содржина и како јазичен
израз.

Посебно внимание треба да се обрне на оформувањето на текстот како завршена
целина. Треба да се пишува за самата тема, да не се шири нејзиниот обем, да не
се за­мени темата со некоја друга тема.
 
Треба да се прави разлика помеѓу исцрпност и завршеност на текстот.

Исцрпноста се состои во опстојна, подробна и целосна расправа за некој проблем.
Есејот по својот карактер (краток текст) не може да претставува так­во исцрпно и
целосно третирање на разгледуваниот проблем.

Текстот треба да биде завршен, т.е. да покажува дека авторот го заокружил
своето излагање.

\section{Цитати и документирање на есејот}

Текстот на есејот не треба да се преоптоварува со цитати (туѓи мисли).
Позначајна е анализата што ќе се направи и оправдувањето на позицијата „за“ или
„против“;

Доколку се користат цитати, тие треба да се стават во навод­ници и да се наведат
точни податоци за авторот, делото и изданието од кои се преземени - да се
документираат.


\section{Редактирање на есејот}

Кога авторот го завршил есејот, треба внимателно да го прочита и да процени каде
и како да го поправи за да го подобри. Најнапред треба да ја проследиме неговата
струк­тура. Тоа значи:

дали се поврзани почетокот, основното изложување и заклучните зборови;

дали создават впечаток на целина;

дали е добро формулирана основната теза;

дали редоследот на аргументите, на тврдењата и при­мерите (или фактите)
           соодветствува на типот на врските помеѓу нив;

дали напишаното е сконцентрирано на темата, има ли замена или отклонување од
предметот на текстот;

има ли повторувања кои не внесуваат дообјаснувања или смисловни нијанси;

како се вградени и распоредени исползуваните типови на текст:   

опишување, раскажување, расудување.

Треба да се обрне внимание и на јазичната конструкција - дали речениците
изразуваат завршена мисла.

Чувството за диалог со замислениот собеседник може да го заведе авторот и да му
се пот­краднат жаргонски изрази, кои во овој случај се недопушт­ливи, бидејќи
есејот е форма на официјална комуникација.

При препрочитувањето на текстот авторот треба да се потруди да открие:
 
дали има непримерено упоребени прашални и извични реченици;
 
        дали се точно напишани термините и имињата;

Треба да се исправат правописните и граматичките грешки.

Текстот на конечната верзија на есејот треба биде исчукан на машина или
испринтан на бела хартија, стандарден фор­мат А4, со 2,5 см. маргина околу
целиот текст на трудот ­лево, десно, горе и долу.

Текстот се пишува само на едната страна од листот, со 12 точки големина на
буквите, со некој од македонските фон­тови (македонска поддршка) и 1.5 проред.

Есејот треба да има насловна страница на која ќе се наведе името на
универзитетот, називот на студиите, целосниот нас­лов на трудот, името на
авторот, името на менторот, време­то и местото (види го примерот на следната
страница).

Пример за насловна страница: 

УНИВЕРЗИТЕТ „СВ. КИРИЛ И МЕТОДИЈ“ − СКОПЈЕ
ФИЛОЗОФСКИ ФАКУЛТЕТЕТ

ГЛОБАЛИЗАЦИЈАТА И НАЦИОНАЛНИОТ ИДЕНТИТЕТ 

- есеј - 

Студент: Александар Николовски
                                       Ментор:  Проф. д-р Ана Димитрова



СКОПЈЕ

Октомври, 2011
\end{document} 

